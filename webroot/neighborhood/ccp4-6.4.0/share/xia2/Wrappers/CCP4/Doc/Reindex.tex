\documentclass[a4paper, 11pt]{article}
\begin{document}
\section{Introduction}

This document describes what is needed to make a wrapper for the CCP4 
program reindex. This is used to map reflection indices from one spacegroup,
most useful when an integrated data set is found to be indexed in the wrong
pointgroup or setting.

\subsection{Background}

The most likely case where this would be used is in response to a run 
of pointless. The possible cases for this are:

\begin{itemize}
\item{Solitary data set - indexing to the correct pointgroup.}
\item{Matching a moving data set to a reference set.}
\item{Correctly assigning the spacegroup in the file header.}
\end{itemize}

Combinations of these are also likely. An example of a reindex command script
is:

{
\tiny
\begin{verbatim}
#!/bin/bash
reindex HKLIN 12287_1_E1_.mtz HKLOUT 12287_1_E1_scaled_reindex.mtz << eof
SYMMETRY 'P 4 2 2'
REINDEX h, k, l
eof
\end{verbatim}
}

The correct symmetry and reindexing operator will come from pointles, though
by specifying a HKLOUT file it is possible to simply use pointless for this.
However, this may be useful when a user has provided the ``correct'' spacegroup
or something.

\section{Use Cases}

\subsection{Use Case 1: Reindexing to a provided spacegroup}

This presumes that you simply want to change the spacegroup label in the 
reflection file but not change the reflection indices. This is correct for
changing from tetragonal spacegroup P4 to say P422. The JCSG dataset 1VPJ
illustrates this well.

\subsection{Use Case 2: Reindexing with a provided reindexing operator}

If you have a spacegroup like P23 where alternative indexings are possible,
it may be necessary to perform the reindexing to ensure that the resulting
reflections match up correctly. One of Ed Mitchell's data sets illustrates 
this well - whichever one is cubic.


\end{document}